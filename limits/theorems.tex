\documentclass{article}
\usepackage{amsmath, amssymb, mathtools}
\usepackage{amsthm}
\usepackage{physics}
\usepackage{fullpage}
\usepackage{titlesec}

\titleformat{\section}{\normalfont\Large\bfseries}{\thesection}{1em}{}

\title{\textbf{Spivak Calculus: Epsilon-Delta Limit Theorems}}
\author{Meet Shah}
\date{}

\newtheorem{theorem}{Theorem}
\newtheorem*{proof*}{Proof}

\begin{document}

\maketitle

\section*{Introduction}
This document contains detailed epsilon-delta proofs of the basic limit theorems as presented in Michael Spivak's ``Calculus.'' T

\section*{Limit Theorems and Proofs}

\begin{theorem}[Limit of Identity Function]
Let $f(x) = x$. Then
\[ \lim_{x \to a} f(x) = a. \]
\end{theorem}

\begin{proof}
Let $\varepsilon > 0$ be given. Choose $\delta = \varepsilon$. Then for all $x$ such that $0 < |x - a| < \delta$, we have:
\[
|f(x) - a| = |x - a| < \delta = \varepsilon.
\]
Hence, $\lim_{x \to a} x = a$.
\end{proof}

\begin{theorem}[Limit of a Constant Function]
Let $f(x) = c$ where $c$ is a constant. Then
\[ \lim_{x \to a} f(x) = c. \]
\end{theorem}

\begin{proof}
Let $\varepsilon > 0$ be given. For all $x$,
\[
|f(x) - c| = |c - c| = 0 < \varepsilon.
\]
So the limit holds for any $\delta > 0$. Thus, $\lim_{x \to a} c = c$.
\end{proof}

\begin{theorem}[Sum of Limits]
If $\lim_{x \to a} f(x) = L$ and $\lim_{x \to a} g(x) = M$, then
\[ \lim_{x \to a} (f(x) + g(x)) = L + M. \]
\end{theorem}

\begin{proof}
Let $\varepsilon > 0$. Since $f(x) \to L$ and $g(x) \to M$, there exist $\delta_1$, $\delta_2 > 0$ such that:
\[
|f(x) - L| < \frac{\varepsilon}{2} \quad \text{if } 0 < |x - a| < \delta_1,
\]
\[
|g(x) - M| < \frac{\varepsilon}{2} \quad \text{if } 0 < |x - a| < \delta_2.
\]
Let $\delta = \min(\delta_1, \delta_2)$. Then for all $x$ such that $0 < |x - a| < \delta$,
\[
|(f(x) + g(x)) - (L + M)| \leq |f(x) - L| + |g(x) - M| < \frac{\varepsilon}{2} + \frac{\varepsilon}{2} = \varepsilon.
\]
Hence, $\lim_{x \to a} (f(x) + g(x)) = L + M$.
\end{proof}

\begin{theorem}[Product of Limits]
If $\lim_{x \to a} f(x) = L$ and $\lim_{x \to a} g(x) = M$, then
\[ \lim_{x \to a} f(x)g(x) = LM. \]
\end{theorem}

\begin{proof}
Let $\varepsilon > 0$. Since $f(x) \to L$, there exists $\delta_1 > 0$ such that
\[
|f(x) - L| < \frac{\varepsilon}{2(|M| + 1)} \quad \text{when } 0 < |x - a| < \delta_1.
\]
Since $g(x) \to M$, there exists $\delta_2 > 0$ such that
\[
|g(x) - M| < 1 \quad \text{when } 0 < |x - a| < \delta_2.
\]
Then $|g(x)| \leq |M| + 1$. Let $\delta = \min(\delta_1, \delta_2)$. Then:
\begin{align*}
|f(x)g(x) - LM| &= |f(x)g(x) - Lg(x) + Lg(x) - LM| \\
&= |g(x)(f(x) - L) + L(g(x) - M)| \\
&\leq |g(x)||f(x) - L| + |L||g(x) - M| \\
&< (|M| + 1)\cdot\frac{\varepsilon}{2(|M| + 1)} + |L|\cdot 1 = \frac{\varepsilon}{2} + \frac{\varepsilon}{2} = \varepsilon.
\end{align*}
\end{proof}

\begin{theorem}[Scalar Multiple]
Let $c$ be a constant. Then
\[ \lim_{x \to a} c\cdot f(x) = cL. \]
\end{theorem}

\begin{proof}
Let $\varepsilon > 0$. Since $\lim f(x) = L$, there exists $\delta > 0$ such that
\[
|f(x) - L| < \frac{\varepsilon}{|c|} \quad \text{if } 0 < |x - a| < \delta.
\]
Then:
\[
|cf(x) - cL| = |c||f(x) - L| < |c| \cdot \frac{\varepsilon}{|c|} = \varepsilon.
\]
\end{proof}

\begin{theorem}[Quotient of Limits]
If $\lim f(x) = L$, $\lim g(x) = M$, and $M \neq 0$, then
\[ \lim_{x \to a} \frac{f(x)}{g(x)} = \frac{L}{M}. \]
\end{theorem}

\begin{proof}
Since $g(x) \to M \neq 0$, there exists $\delta_1 > 0$ such that $|g(x) - M| < \frac{|M|}{2} \Rightarrow |g(x)| > \frac{|M|}{2}$. Also, since $f(x) \to L$, there exists $\delta_2 > 0$ such that $|f(x) - L| < \varepsilon \cdot \frac{|M|}{4}$. 

Now:
\begin{align*}
\left|\frac{f(x)}{g(x)} - \frac{L}{M}\right| &= \left|\frac{f(x)M - Lg(x)}{g(x)M}\right| \\
&= \left|\frac{M(f(x) - L) + L(M - g(x))}{g(x)M}\right| \\
&\leq \frac{|M||f(x) - L| + |L||g(x) - M|}{|g(x)||M|}
\end{align*}
Because $|g(x)| > |M|/2$, the denominator is bounded away from 0. With appropriate choice of $\delta$, we can ensure the total is less than $\varepsilon$.
\end{proof}

\begin{theorem}[Absolute Value of Limit]
If $\lim f(x) = L$, then
\[ \lim |f(x)| = |L|. \]
\end{theorem}

\begin{proof}
We use the inequality:
\[
\left||f(x)| - |L|\right| \leq |f(x) - L|.
\]
Given $\varepsilon > 0$, choose $\delta > 0$ such that $|f(x) - L| < \varepsilon$ when $0 < |x - a| < \delta$. Then
\[
\left||f(x)| - |L|\right| < \varepsilon,
\]
which proves the result.
\end{proof}

\end{document}

